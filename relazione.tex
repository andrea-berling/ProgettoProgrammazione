\documentclass[10pt]{report}

\usepackage[utf8]{inputenc}
\usepackage[italian]{babel}
\usepackage{enumitem,hyperref}
\hypersetup{colorlinks=true,allcolors=black}
\RequirePackage[l2tabu, orthodox]{nag}
\title{Relazione del gioco}
\author{Andrea Berlingieri, 
	Giacomo Puligheddu, 
	Riccardo Bellelli}
\date{}

\begin{document}
\maketitle
\tableofcontents

\chapter{Scelte implementative}

\section{La mappa}

La mappa è generata proceduralmente. Si parte da un'area iniziale che viene divisa in due, dopodichè si
dividono le due aree ottenute, poi le quattro ottenute ora, e così via, fino a raggiungere un numero di aree
che non si intersecano pari al numero di stanze richieste. Quando il numero delle stanze diventa grande è
possibile che l'algoritmo di divisione non riesca a terminare con un numero di aree uguale a quello richiesto.
Per questo motivo all'algoritmo sono permessi un numero fisso di tentativi. La dimensione delle aree dopo la
suddivisione è decisa casualmente, tenendo conto che la divisione può creare aree la cui dimensione minima
è un terzo dell'area iniziale e la cui dimensione massima è due terzi dell'area iniziale.  
In ognuna di queste aree viene generata una stanza interna all'area.

Per la generazione di una stanza si sceglie casualmente un punto all'interno dell'area tale da determinare una
stanza di area massima uguale a quella dell'area di appartenzenza meno il bordo e di dimensioni minime 5x5.
Successivamente sono scelte, in modo casuale ma rispettando i limiti sulle dimensioni, l'altezza e la
larghezza della stanza. Le dimensioni massime di una stanza sono 20x20.

Dopo che le stanze sono state generate, queste vengono collegate mediante corridoi. Un corridoio di
collegamento tra una stanza e l'altra è generato mediante la ricerca di un cammino minimo tra un punto
(casuale) della prima stanza ed uno dell'altra in un grafo che contiene tutti i punti dell'area iniziale meno
il bordo e i punti delle stanze. 

Una volta che le stanze e i corridoi sono stati generati vengono piazzati sulla mappa, che mantiene una
griglia di tiles inizializzate a WALL. 

\section{Livelli}

Un livello è definito dal suo numero, dalla sua mappa, dai suoi mostri e dai suoi oggetti. Questi sono mantenuti in
delle tabelle hash che come chiave utilizzano l'id dell'oggetto o mostro. In questo modo ogni volta che si accede allo
stesso livello la mappa, la disposizione dei mostri e degli oggetti rimangono uguali a quando si abbandona il livello,
passando ad uno superiore o inferiore con le scale.

Per passare da un livello all'altro si utilizzano delle scale. Si hanno sia scale per salire di livello, che scale per
tornare indietro di un livello, fatta ovviamente eccezione per il primo livello.

I livelli vengono mantenuti in una lista ed il passaggio da un livello ad un altro corrisponde all'incremento o
decremento di un iteratore che punta al livello corrente. Se la lista ``finisce'' un nuovo livello viene creato.

\section{Grafica}

Per la parte grafica del gioco si è utilizzata la libreria ncurses insieme alle librerie aggiuntive menu e
panel, che definiscono alcuni ``widget'' per ncurses che ne rendono l'utilizzo pratico più agevole. Le strutture dati che
rappresentano le finestre ed i menù in ncurses sono state ``racchiuse'' in dei wrapper ad oggetti. In questo modo la
creazione di una finestra o di un menù è resa più agevole tramite l'utilizzo di appositi costruttori, e una volta che
la loro ``vita utile'' termina la memoria occupata viene liberata automaticamente tramite appositi distruttori. Inoltre le
azioni di uso comune (come la stampa di una linea, la pulizia di una finestra, la scelta di un'opzione di un menu,
ecc.) sono eseguite mediante apposite funzioni membro.

\section{La battaglia}

Lo svolgimento della battaglia è mostrato al di sotto della mappa, in una finestra nella quale appaiono le statistiche del nemico e le perdite durate gli attacchi da parte di entrambi gli avversari, tenendo aggiornati i parametri del player e del mostro. Inoltre se viene scelto l'inventario verrà aggiornata la finestra degli equipaggiamenti così da permette al giocatore di visionare che parti dell'armatura indossa.

L'inizio della battaglia è caratterizzato dall'attacco del personaggio principale, il quale può scegliere tra tre differenti opzioni:
\begin{enumerate}

    \item \textbf{Attacco}: il player attacca l'avversario, può scegliere tra due diverse mosse:
 	\begin{itemize}

	     \item La prima è l'attacco normale, che toglie all'avversario punti vita in base all'attacco del player e
	     alla difesa del nemico, ed in questo caso è presente una scelta randomica, che determina se l'attacco sarà
	     oppure no critico; se così fosse, il PG ottiene attacco raddoppiato, aumentando il danno inflitto al
	     mostro;
	     \item La seconda riguarda la mossa speciale, che differisce in base al personaggio: Gaudenzio sfrutta la
	     rigenerazione (guadagna LP in base al suo livello), Peppino utilizza un attacco magico (attraverso i punti
	     mana raddoppia l'attacco) e Badore sfrutta l'attacco furtivo (può triplicare i danni al nemico, ma se viene
	     scoperto perde il turno e il mostro fa due attacchi).
	\end{itemize}
Se non viene scelta nessuna delle due opzioni, il player perde un turno per aver sbagliato a digitare, lasciando l'attacco all'avversario.

    \item \textbf{Inventario}: si può consultare l'inventario e guardare gli oggetti, si possono equipaggiare, disequipaggiare, droppare o utilizzare se sono consumabili. Tuttavia si perderà un turno di gioco, consentendo al nemico di attaccare.

    \item \textbf{Corruzione}: invece che combattere il nemico, il player può affidarsi a una scelta randomica e e nel qual caso vada a
    buon fine, può far si che il mostro se ne vada, pagando un determinato numero di monete (Cucuzze).
\end{enumerate}

Il mostro possiede solo l'attacco normale e il suo colpo potrebbe fallire, mediante una scelta randomica.
La battaglia tra player e avversario prosegue finché uno dei due non riesce a sopraffare
l'altro. Se il player vince, guadagna una somma di denaro che dipenderà dal tipo di mostro affrontato.
Nel caso opposto, cioè se vince il nemico, il gioco finisce in quanto si è stati sconfitti.

\section{I personaggi}

La classe personaggi contiene tutte le statistiche relative al PG, compreso il nome e la posizione sulla
mappa. 
Per quanto riguarda l'inventario, questo fa parte della classe del pg ed è implementato come un 
unordered set, ovvero come un insieme di item senza un ordine particolare. Tale scelta consente di avere
dei tempi di accesso agli item molto bassi rispetto a un'altra struttura dati. 
Oltre all'inventario è presente un vettore chiamato equipaggiamento che contiene, appunto,
gli oggetti che il pg ha equipaggiato.

Oltre ai metodi set, get, pickItem, dropItem.. questa classe contiene anche un metodo LVLup che si occupa
di far salire le statistiche del PG, in relazione al PG scelto: il guerriero non otterrà mana e avrà un
incremento di punti vita mentre il mago avrà più mana e meno punti vita; il ladro otterrà una maggiore
fortuna. I personaggi salgono di livello man mano che si scoprono nuovi livelli: se ci troviamo per la prima volta al
livello n, il PG passerà da livello n-1 a livello n. Questo permette un bilanciamento tra
le statistiche del PG e quelle dei mostri.
Il metodo showInventory è all'interno della classe personaggi perchè questo ha contribuito a una 
realizzazione più semplice (dato che così può accedere ai campi del PG). Tale metodo gestisce
l'interazione dell'utente con l'inventario del suo personaggio.
Infine si è rivelato utile utilizzare una funzione (suffix) per poter decidere come riferirsi a un
determinato oggetto: alcuni item sono maschili, altri femminili e altri al plurale (stivali, spada,..).

\section{Gli oggetti}

La classe Item ha al suo interno tutti i campi relativi alle statistiche modificate, alla posizione sulla
mappa, al simbolo con la quale verrà visualizzato e al fatto se l'item è visibile o meno.

Due sono i campi fondamentali: id e type.
Il campo id consente di identificare univocamente un Item, ovvero di distinguere più oggetti tra loro,
anche con lo stesso nome. Questa scelta è stata necessaria perchè potrebbero comparire più 
oggetti identici sulla mappa. Per quanto riguarda il campo type, questo è l'indice con la quali gli item
verranno inseriti nel vettore equipaggiamento; un item di tipo elmo avrà type uguale a
0, un item di tipo corazza avrà type uguale a 1 e così via. I consumabili sono di type 5 (il vettore 
equipaggiamento arriva fino all'indice 4). 

È stato necessario confrontare gli oggetti tra loro, pertanto sono stati creati due metodi per poter
decidere se due item sono uguali o diversi; sono uguali se hanno lo stesso id, diversi altrimenti.

\section{Inventario}
Per rappresentare l'inventario viene creata una finestra contenente un menù che permette di spostarsi tra i vari Item; in una finestra laterale vengono mostrate le statistiche e al di sotto di queste ci sono le opzioni disponibili.

L'inventario si può consultare sia durante la perlustrazione della mappa, sia quando 
si sta combattendo contro un mostro. 
Se l'oggetto è un consumabile, una volta usato viene cancellato dall'inventario; 
Se l'oggetto è equipaggiabile, può essere indossato o rimosso. Gli spazi che non contengono oggetti sono vuoti e vengono
contrassegnati dalla scritta ``-VUOTO-'' così da poter monitorare le capacità dell'inventario. 

Per tutti gli oggetti è inoltre disponibile l'opzione di drop, che consiste nel lasciare l'oggetto, liberando così uno slot.

\section{I mostri}

La classe Monster contiene i campi relativi alle statistiche, al livello, alla posizione e al nome.
Ha poi altri campi tra cui id (che funziona come per gli item) e symbol che, in base al mostro, indica
il carattere con cui quel nemico verrà stampato sulla mappa.

Anche qui sono presenti i vari metodi set e get e i metodi necessari per confrontare due mostri e vedere
se sono uguali oppure no (in base all'id).

Il metodo moveMonster gestisce lo spostamento dei mostri sulla mappa. Prevede due casi: 
casuale oppure inseguimento del PG (se nella stessa stanza). Quando il PG lascia la stanza i mostri iniziano a muoversi
casualmente, senza uscire da questa. La scelta è stata necessaria perchè altrimenti, 
il PG sarebbe sempre costretto ad affrontare i mostri che lo bloccherebbero nei corridoi.	

\section{Lo Shop}

Ogni volta che si sale di livello il giocatore ha la possibilità di acquistare uno tra tre oggetti per 
il PG. Una volta comprato un oggetto (o dichiarato lo stato di povertà) il menù si chiude e il nuovo livello viene
stampato. Affinchè non vengano presentati tre oggetti identici, è stata utilizzata una funzione (generateKPermutation)
per generare una disposizione di 3 oggetti. Per garantire un buon bilanciamento del gioco, i tre oggetti hanno prezzo
relativo alla loro rarità, ovvero all'indice con cui sono stati inseriti nel vettore ItemsSet dalla funzione
retrieveItems.

\end{document}
